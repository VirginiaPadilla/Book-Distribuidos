\setchapterstyle{lines}
\labch{appendix}
%%\blinddocument

\chapter{Caso de Estudio: World Wide Web }
\label{ch:caso-estudio}
El caso de estudio que se presenta est\'a tomado de Couloris \sidecite{Coulouris2011}. En este caso de estudio se presentan las tecnolog\'ias fundamentales sobre la cual esta\'an construidos los sistemas distribuidos.
La Web es un sistema abierto, con las siguientes caracter\'isticas: 
\begin{enumerate}
	\item Su funcionamiento est\'a 	basado en estándares de comunicación y contenido o en documentos que se  	publican e implementan libremente. Por ejemplo, hay muchos tipos de navegador,  implementado en varias plataformas; as\'i como   muchas implementaciones de servidores web. Cualquier navegador compatible puede recuperar recursos de cualquier servidor compatible. As\'i que, los usuarios tienen acceso a los navegadores desde   los dispositivos que utilizan,
	desde teléfonos móviles hasta computadoras de escritorio.
	
	\item La Web está abierta con respecto a los tipos de \gls{recursos} que se pueden publicar y compartir. Si alguien inventa, digamos, un nuevo formato de almacenamiento de imágenes, entonces  las imágenes en este formato se pueden publicar inmediatamente en la Web. Los usuarios necesitan un medio de ver imágenes en este nuevo formato. Los navegadores están diseñados para adaptarse a nuevas funcionalidades de presentación de contenido en forma de aplicaciones auxiliares y complementos o \textit{plug-ins}.
	
\end{enumerate}

La Web se basa en  componentes tecnológicos que son  estándares de la W3C, \sidecite{W3CConsultado2021} y que a continuaci\'on se describen.

\begin{description}
	\item[{HTML}].  \index{HTML} El lenguaje de marcado de hipertexto, \gls{HTML}   se utiliza para especificar el texto e imágenes que componen el contenido de una página web, y  cómo se colocan y formatean,  para su presentación al usuario. Una página web contiene dichos elementos estructurados como títulos, párrafos, tablas e imágenes. \textit{HTML} también se utiliza para especificar enlaces y  recursos que están asociados a ellos. 
	\marginnote[-1cm]{
		\begin{kaobox}[frametitle=HTML ]
			Conozca m\'as de URL visitando este sitio  \url{https://www.w3.org/html/}
		\end{kaobox}
	}
	
	A continuación, se muestra un fragmento típico de texto \textit{HTML}, ver figura \ref{fig:Cou-Fig2}:
	
	
	\begin{figure}%
		\includegraphics{Coulouris-Fig2.png}
		\caption{Texto en HTML. Tomado de \CO }
		\label{fig:Cou-Fig2}
	\end{figure}
	
	Este texto \textit{HTML} se almacena en un archivo al que puede acceder un servidor web, por ejemplo  el archivo \textit{earth.html}. Un navegador recupera el contenido de este archivo de un servidor web, en este caso un servidor en una computadora llamada \textit{www.cdk5.net}. El navegador lee el contenido devuelto por 	el servidor y lo convierte en texto formateado e imágenes distribuidas en una página web de una manera adecuada. Solo el navegador, no el servidor, interpreta el texto \textit{HTML}. 
	
	Pero  el servidor informa al navegador del tipo de contenido que está devolviendo, para distinguirlo de, por ejemplo, un documento en formato de documento portátil. El servidor puede inferir el contenido escrito cuando reconoce la extensión de nombre de archivo \textit{html}.
	
	La línea 1 del ejemplo \ref{fig:Cou-Fig2}  identifica un archivo que contiene una imagen  su presentación, mediante su \textit{URL}.
	Línea 2 a la  5 son directivas para comenzar y finalizar un párrafo, respectivamente. 
	En las líneas 3 y 4 contienen texto que se mostrará en la página web en el formato de párrafo estándar.
	La línea 4 especifica un enlace en la página web. Contiene la palabra \textit{Moon}  rodeada 	por dos etiquetas HTML relacionada, $<A$ $HREF...>$  y {$</A>$}. El texto entre estas etiquetas es lo que
	aparece en el enlace tal como se presenta en la página web. 
	
	El navegador registra la asociación entre el texto que se muestra en el enlace y el URL contenido en la etiqueta {$<A$ $HREF...>$} - en este caso:
	
	\begin{kaobox}[frametitle= Enlace  URL]
		Enlace $--->$ \href{http://www.cdk5.net/WebExample/moon.html}{Moon}	
	\end{kaobox}
	
	Cuando el usuario hace clic en el texto, el navegador recupera el recurso identificado por el URL correspondiente y la presenta al usuario. En el ejemplo, el recurso es un archivo \textit{HTML}  que especifica una página web sobre la Luna.
	
	\item[{URL}] \index{URL} Los navegadores examinan las \gls{URL}	  para acceder a las correspondientes recursos. A veces, el usuario escribe una \textit{URL} en el navegador, o el  navegador busca la \textit{URL} correspondiente cuando el usuario hace clic en un enlace o selecciona uno de sus marcadores o \textit{bookmarks}.
	
	\marginnote{
		\begin{kaobox}[frametitle=URL ]
			Conozca m\'as de URL visitando este sitio
			\href{https://www.w3.org/TR/url/}{URL}  
		\end{kaobox}
	}
	
	
	Cada \textit{URL}, en su forma completa y absoluta, tiene dos componentes de nivel superior:
	
	
	
	\begin{kaobox}[frametitle=Componentes URL]
		scheme : scheme-specific-identifier
	\end{kaobox}
	
	El primer componente declara qué tipo de \textit{URL} es. Los \textit{URL} son necesarios para identificar una variedad de recursos. Por ejemplo, \textit{mailto: joe@anISP.net}	identifica la dirección de correo electrónico de un usuario; \textit{ftp://ftp.downloadIt.com/software/aProg.exe} identifica
	un archivo que se recuperará utilizando el Protocolo de transferencia de archivos (FTP) en lugar del	protocolo HTTP.  
	
	Las \texttt{URL HTTP} son las más utilizadas para acceder a los recursos utilizando el protocolo estándar HTTP. 
	Una \texttt{URL HTTP} tiene dos funciones principales: identificar qué servidor web mantiene el recurso e identificar cuál de los recursos de ese servidor es necesario.
	
	La Figura \ref{fig:WebServer-Exa} muestra tres navegadores que emiten solicitudes de recursos que son administrados por tres servidores web. El navegador superior está emitiendo una consulta a un motor de búsqueda. El navegador del medio requiere la página predeterminada de otro sitio web. El navegador inferior requiere una p\'agina web  que se especifica en su totalidad, incluido un nombre de ruta relativo al servidor.  
	
	\index{DNS}	En general, las \texttt{URL HTTP} tienen el siguiente formato, ver figura \ref{fig:Cou-Fig3}:
	
	\begin{figure}%
		\includegraphics{Coulouris-Fig3.png}
		\caption{ Formato de los URL. Tomado de \CO }
		\label{fig:Cou-Fig3}
	\end{figure}
	
	donde los elementos entre corchetes son opcionales. Una\texttt{ URL HTTP} completa siempre comienza con cadena \textit{http: //} seguida de un nombre de servidor, expresado como un sistema de nombres de dominio (DNS). Opcionalmente, el nombre \gls{DNS} del servidor va seguido del número del puerto en el que el servidor escucha las solicitudes,  el puerto 80 por 	defecto. Luego viene un nombre de ruta opcional del recurso del servidor. Si esto está ausente entonces se requiere la página web predeterminada del servidor. Finalmente, la URL termina opcionalmente en una consulta. 
	
	
	\begin{figure}
		\includegraphics{Coulouris-WebServer.png}
		\caption{Ejemplo de Web Server y Web Browser. Tomado de \CO}
		\label{fig:WebServer-Exa}
	\end{figure}
	
	\item[{HTTP}] \index{HTTP} El Protocolo de transferencia de hipertexto, \gls{HTTP} \footnote{Amplie su conocimiento visitando este sitio \href{https://www.w3schools.com/whatis/whatis_http.asp} {What is HTTP}  } presenta estas caracter\'isticas:
	
	\begin{itemize}
		\item \texttt{Interacciones de solicitud-respuesta}.  \textit{HTTP} es un \gls{protocolo solicitud-respuesta}.  \textit{HTTP} define un pequeño conjunto de operaciones o métodos que pueden 	realizarse en un recurso. Los más comunes son \textit{GET}, para recuperar datos del recurso y \textit{POST}, para proporcionar datos al recurso.
		
		\item \texttt{Tipos de contenido}. Los navegadores no son necesariamente capaces de manejar todo tipo de
		contenido. Cuando un navegador realiza una solicitud, incluye una lista de los tipos de contenido que prefiere, por ejemplo, en principio, puede mostrar imágenes en formato \textit{GIF} 		
		pero no en formato \textit{JPEG}. El servidor puede tener esto en cuenta cuando devuelve contenido al navegador. El servidor incluye el tipo de contenido en el mensaje de respuesta para que el navegador sepa cómo procesarlo. El conjunto de las acciones que realizará un navegador para un tipo de contenido determinado son configurables.
		
		\item 	\texttt{Un recurso por solicitud}. Los clientes especifican un recurso por cada  solicitud \textit{HTTP}. Por ejemplo, si una p\'agina web  contiene nueve imágenes,  entonces el navegador emitirá un total de diez solicitudes por separado para la obtención de todo el contenido de la página. Los navegadores suelen hacer varias solicitudes al mismo tiempo, para reducir la demora general para el usuario.
		
		\item \texttt{Control de acceso simple}.  Ccualquier usuario con conectividad de red a un servidor Web puede acceder a cualquiera de sus recursos publicados de forma predeterminada. Si los usuarios desean restringir el acceso a un recurso,  pueden configurar el servidor para que emita una alerta  a cualquier cliente que lo solicita. El usuario correspondiente debe demostrar que tiene derecho a acceder el recurso, por ejemplo, escribiendo una contraseña.
		
	\end{itemize}
	
	%%	\item[{P\'aginas din\'amicas}]  \index{p\'aginas din\'amicas} Gran parte de la experiencia de los usuarios en la Web es el de interactuar con los servicios en lugar de recuperar datos. Por ejemplo, cuando se compra un artículo en una tienda en línea, el usuario  completa un \gls{formulario web} para proporcionar datos personales o para especificar  lo que se desea comprar.  Cuando el usuario envía el formulario, el navegador envía una solicitud \textit{HTTP} a un servidor web, que contiene los valores que el usuario ha ingresado.
	
	%%%	Dado que el resultado de la solicitud depende de la entrada del usuario, el servidor debe procesar la entrada del usuario. Por lo tanto, la \textit{URL} o su componente inicial designa un 	programa en el servidor. 	Si la entrada del usuario es un conjunto  pequeño de 	parámetros,  se envía como el componente de consulta de la \textit{URL}, utilizando m\'etodo \textit{GET} o alternativamente, se envía como datos adicionales en la solicitud utilizando el m\'etodo \textit{POST}.
	
	%%	Por ejemplo, una solicitud que contiene la siguiente URL invoca un programa llamado búsqueda en \textit{ www.google.com} y especifica una cadena de consulta de \textit{mamul}: 	http://www.google.com/search?q=mamul. 	Ese programa de búsqueda produce texto HTML como salida, y el usuario verá un lista de páginas que contienen la palabra \textit{mamul}.  El servidor devuelve el texto HTML que genera el programa  aunque lo haya recuperado de un archivo. En otras palabras, la diferencia entre el contenido estático  obtenido de un archivo y el contenido  que se genera dinámicamente es transparente para 	el navegador.
	
	\item[{Servicios Web}] \index{servicios!Web} Otra manera de usar la web es a tr\'aves de los  programas que son clientes de un servicio Web. Sin embargo, \textit{HTML} es inadecuado para la interoperación programática.
	
	
	\marginnote {
		\begin{kaobox}[frametitle=HTML  limitado ]
			Limitado en el sentido de que no es extensible a aplicaciones más allá de la exploración de información,  tiene un conjunto estático de estructuras, y están vinculados con la forma en que los datos se   presentan a los usuarios.
		\end{kaobox}
	}
	
	
	El lenguaje de marcado extensible, \gls{XML}  ha sido diseñado como una forma de representar datos estandarizados y estructurados,  en formatos espec\'ificos de la aplicaci\'on. Los datos expresados en  ese lenguaje son portables entre aplicaciones ya que es autodescriptivo.
	
	\marginnote [2cm] {
		\begin{kaobox}[frametitle=XML]
			Conozca m\'as de XML visitando este sitio  \href{https://www.w3schools.com/xml/} {XML}
		\end{kaobox}
	}
	
	En el protocolo \textit{HTTP}, los datos \textit{XML} pueden ser transmitidos por las operaciones \textit{POST} y \textit{GET}.   Por ejemplo, en la tienda de \textit{amazon.com}, las operaciones del servicio web incluyen una para pedir un libro y otra para comprobar
	el estado actual de un pedido.
	
	%	\marginnote {
		\begin{kaobox}[frametitle=REST ]
			REST   \cite{Fielding2000} adopta este enfoque sobre la base de su extensibilidad: cada recurso en la Web tiene una URL y 	responde al mismo conjunto de operaciones, aunque el procesamiento de las operaciones puede varíar mucho de un recurso a otro.
		\end{kaobox}
		%	} 
	
	Cualquier operación sobre un recurso puede ser invocado usando uno de los métodos \textit{POST} o \textit{GET}, con contenido estructurado usado para especificar los parámetros, resultados y respuestas de error de la operación. 
	
\end{description}

%%%%%
